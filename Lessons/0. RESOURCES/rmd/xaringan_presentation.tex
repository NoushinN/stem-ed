\PassOptionsToPackage{unicode=true}{hyperref} % options for packages loaded elsewhere
\PassOptionsToPackage{hyphens}{url}
%
\documentclass[]{article}
\usepackage{lmodern}
\usepackage{amssymb,amsmath}
\usepackage{ifxetex,ifluatex}
\usepackage{fixltx2e} % provides \textsubscript
\ifnum 0\ifxetex 1\fi\ifluatex 1\fi=0 % if pdftex
  \usepackage[T1]{fontenc}
  \usepackage[utf8]{inputenc}
  \usepackage{textcomp} % provides euro and other symbols
\else % if luatex or xelatex
  \usepackage{unicode-math}
  \defaultfontfeatures{Ligatures=TeX,Scale=MatchLowercase}
\fi
% use upquote if available, for straight quotes in verbatim environments
\IfFileExists{upquote.sty}{\usepackage{upquote}}{}
% use microtype if available
\IfFileExists{microtype.sty}{%
\usepackage[]{microtype}
\UseMicrotypeSet[protrusion]{basicmath} % disable protrusion for tt fonts
}{}
\IfFileExists{parskip.sty}{%
\usepackage{parskip}
}{% else
\setlength{\parindent}{0pt}
\setlength{\parskip}{6pt plus 2pt minus 1pt}
}
\usepackage{hyperref}
\hypersetup{
            pdftitle={Xaringan Presentation},
            pdfauthor={Noushin Nabavi},
            pdfborder={0 0 0},
            breaklinks=true}
\urlstyle{same}  % don't use monospace font for urls
\usepackage[margin=1in]{geometry}
\usepackage{graphicx,grffile}
\makeatletter
\def\maxwidth{\ifdim\Gin@nat@width>\linewidth\linewidth\else\Gin@nat@width\fi}
\def\maxheight{\ifdim\Gin@nat@height>\textheight\textheight\else\Gin@nat@height\fi}
\makeatother
% Scale images if necessary, so that they will not overflow the page
% margins by default, and it is still possible to overwrite the defaults
% using explicit options in \includegraphics[width, height, ...]{}
\setkeys{Gin}{width=\maxwidth,height=\maxheight,keepaspectratio}
\setlength{\emergencystretch}{3em}  % prevent overfull lines
\providecommand{\tightlist}{%
  \setlength{\itemsep}{0pt}\setlength{\parskip}{0pt}}
\setcounter{secnumdepth}{0}
% Redefines (sub)paragraphs to behave more like sections
\ifx\paragraph\undefined\else
\let\oldparagraph\paragraph
\renewcommand{\paragraph}[1]{\oldparagraph{#1}\mbox{}}
\fi
\ifx\subparagraph\undefined\else
\let\oldsubparagraph\subparagraph
\renewcommand{\subparagraph}[1]{\oldsubparagraph{#1}\mbox{}}
\fi

% set default figure placement to htbp
\makeatletter
\def\fps@figure{htbp}
\makeatother

\usepackage{etoolbox}
\makeatletter
\providecommand{\subtitle}[1]{% add subtitle to \maketitle
  \apptocmd{\@title}{\par {\large #1 \par}}{}{}
}
\makeatother

\title{Xaringan Presentation}
\providecommand{\subtitle}[1]{}
\subtitle{⚔with xaringan and a twist of hygge}
\author{Noushin Nabavi}
\date{2019/07/19}

\begin{document}
\maketitle

class: center, middle

\hypertarget{hello}{%
\section{hello}\label{hello}}

\hypertarget{heres-a-presentaion}{%
\subsubsection{here's a presentaion}\label{heres-a-presentaion}}

type or insert images, ec. \emph{`where do we go from here'}

\hypertarget{coloured-content-boxes}{%
\subsection{Coloured content boxes}\label{coloured-content-boxes}}

Use \texttt{.content-box-blue} (or gray/grey, army, green, purple, red,
or yellow) to produce a box with coloured background. Size depends on
content.

\texttt{.content-box-blue{[}I\ feel\ blue{]}} yields

.content-box-blue{[}I feel blue{]}

Wrap in \texttt{.full-width} to expand the width

.full-width{[}.content-box-red{[}I feel even more blue{]}{]}

If you have content in columns then you get

.pull-left{[}.full-width{[}.content-box-yellow{[}\textbf{WARNING} Look
out for minons or bananas{]}{]}{]}
.pull-right{[}.full-width{[}.content-box-yellow{[}The box to the left
was created using
\texttt{.pull-left{[}.full-width{[}.content-box-yellow{[}{]}{]}{]}}{]}{]}{]}

\begin{center}\rule{0.5\linewidth}{0.5pt}\end{center}

\hypertarget{fancy-picture-includes}{%
\subsection{Fancy picture includes}\label{fancy-picture-includes}}

.pull-left{[} Original:

\includegraphics[width=0.8\linewidth]{https://www.worldtravelguide.net/wp-content/uploads/2017/04/Think-Denmark-Copenhagen-587892190-SeanPavonePhoto-copy}

Add \texttt{.polaroid}

.polaroid{[}

\includegraphics[width=0.8\linewidth]{https://www.worldtravelguide.net/wp-content/uploads/2017/04/Think-Denmark-Copenhagen-587892190-SeanPavonePhoto-copy}
{]} {]} .pull-right{[}

Rotated slightly:

.rotate-right{[}

\includegraphics[width=0.8\linewidth]{https://www.worldtravelguide.net/wp-content/uploads/2017/04/Think-Denmark-Copenhagen-587892190-SeanPavonePhoto-copy}
{]}

Add \texttt{.blur}

.blur{[}

\includegraphics[width=0.8\linewidth]{https://www.worldtravelguide.net/wp-content/uploads/2017/04/Think-Denmark-Copenhagen-587892190-SeanPavonePhoto-copy}
{]} {]}

\begin{center}\rule{0.5\linewidth}{0.5pt}\end{center}

\hypertarget{modifying-text}{%
\section{Modifying text}\label{modifying-text}}

.pull-left{[}

\hypertarget{font-sizes}{%
\subsection{Font sizes}\label{font-sizes}}

This is normal size ( \(\LaTeX\)-friendly terms)

.Large{[}Large{]}

.large{[}large{]}

.small{[}small{]}

.footnotesize{[}footnotesize{]}

.scriptsize{[}scriptsize{]}

.tiny{[}tiny{]}

{]}

.pull-right{[}

\hypertarget{text-color}{%
\subsection{Text color}\label{text-color}}

.black{[}black{]}

.red{[}red{]}

.blue{[}blue{]}

.green{[}green{]}, .yellow{[}yellow{]}, .orange{[}orange{]},
.purple{[}purple{]}, .gray{[}gray or grey{]}

You can also use \texttt{.bold{[}{]}} or \texttt{.bolder{[}{]}} to
emphasize text

This is .bold{[}bold{]}, this is .bolder{[}bolder{]} and this is regular
markdown \textbf{double-star bold} (visible differences depend on the
font)

{]}

\begin{center}\rule{0.5\linewidth}{0.5pt}\end{center}

\hypertarget{stacking-fancy-picture-options}{%
\subsection{Stacking fancy picture
options}\label{stacking-fancy-picture-options}}

.pull-left{[} Add \texttt{.opacity}

.opacity{[}

\includegraphics[width=0.8\linewidth]{https://www.worldtravelguide.net/wp-content/uploads/2017/04/Think-Denmark-Copenhagen-587892190-SeanPavonePhoto-copy}
{]}

Stack \texttt{.blur} and \texttt{.opacity}

.blur{[}.opacity{[}

\includegraphics[width=0.8\linewidth]{https://www.worldtravelguide.net/wp-content/uploads/2017/04/Think-Denmark-Copenhagen-587892190-SeanPavonePhoto-copy}
{]}{]} {]} .pull-right{[}

Convert to \texttt{.grayscale} (oh \ldots{} and rotate just for s'n'g):

.rotate-left{[} .grayscale{[}

\includegraphics[width=0.8\linewidth]{https://www.worldtravelguide.net/wp-content/uploads/2017/04/Think-Denmark-Copenhagen-587892190-SeanPavonePhoto-copy}
{]}{]}

Add \texttt{.shadow}

.shadow{[}

\includegraphics[width=1\linewidth]{https://beautifulenvironments.files.wordpress.com/2017/12/twinkly-lights}
{]} {]}

\end{document}
